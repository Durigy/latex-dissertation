\chapter*{Abstract}

Students are forced to use many different applications on a daily bases for their studies, some even for the exact same purpose. This dissertation aims to take the first steps at creating an application which brings together some of those features. It also adds a social networking aspect alongside the learning with an aim to connect students and tutors together to lower the barrier of communication.

An application was created that combined a social media platform similar to WhatsApp or Discord and a VLE (Virtual Learning Environment) similar to Blackboard Learn or Moodle. 

The application was successful in creating a platform for students and tutors to communicate and share resources such as PDFs and images/screenshots, but more work needs to be done on improving how resources are organised for modules. The application allows students to access module information, ask questions, and contact other users in one place designed for students. Searching for elements in the application, such as a question, is not possible in the current prototype of the application.
Due to time constraints, the final results of effectiveness and design satisfaction and based on the writes perspective, and experience as a university student of three years.

More work, research and studies are needed to determine the effectiveness for the general population of students.