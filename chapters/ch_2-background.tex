\chapter{Background}

\section{LMS/VLE}
A Learning Management System (LMS) is an application (typically accessible online) used to manage the content of courses for employees (students) in a workplace taking a course, and content can be created by people teaching employees within a company. The LMS allows students to log in to the application, see the new content and use discussion boards if the tutor allows.

A Virtual Learning Environment (VLE) is more focused on taking the concept of an LMS but adding and improving the way in which the content is delivered to students and, more importantly, how a student consumes and interacts with the content and their peers. A VLE is aimed towards universities and other teaching institutions where engagement with the content and other peers is more important. The main idea is to bring the 'on campus' experience (or a version of it) to an online environment.

These applications all have one thing in common, they are built and aimed at the tutor/institution/organisation who deliver the content, not at the students using the systems for their learning. Students don't have permission to join a module/course or view the content available. Tutors can't add a student to the module themselves either; they must ask their school office/department to add the student.

According to neilmosley.com \cite{neilmosley_VLEs}, the split across universities for which VLEs are used in the UK shows Moodle as the highest with 40\% market share, Blackboard dropping down to 28\% and Canvas in third with 17\%.

\subsection{Competition}

\subsubsection{Learning Central}

Blackboard Learn \cite{blackboard_learn} is used by many universities in the UK, including Cardiff University and 40\% of other Russell Group universities\cite{neilmosley_VLEs}. Cardiff University has even re-branded its instance of the application to Learning Central (LC).

Blackboard Learn has many features, such as modules/courses, a calendar (not used by Cardiff University), assessment submission, assessment mark tracking, and video conferencing similar to Zoom.

For more information, see a video demo \cite{blackboard_learn_demo}.

\subsubsection{Moodle}

Moodle \cite{moodle} holds the largest share of university users at 40\% \cite{neilmosley_VLEs} in the UK. Moodle is an open-source application which can be self-hosted on a university's own infrastructure or with the option to pay for Moodle to host it in their cloud for a University.

Demo of Moodle \cite{moodle_demo_1}\cite{moodle_demo_2}\cite{moodle_demo_3}.

\subsubsection{D2L Brightspace}

D2L is designed for companies; the social communication features (live chat) are used during lectures only as  the companies that they are targeting (medium to large enterprises) likely already have a communication system such as Slack or MS Teams 

demo videos of D2L Brightspace \cite{Brightspace_demo_1}\cite{Brightspace_demo_2}\cite{Brightspace_demo_3}.

\subsubsection{Google Classroom}

Google Classroom is free for schools. Typically used by primary and secondary schools. Links in/utilises other Google products i.e Google Forms. Must use a Google account/Google Workspace created by the school. Google Classroom is very barebones and left up to teachers to organise and choose how they display everything. Things can get cluttered and unorganised too.

Demo videos of Google Classroom \cite{google_classroom_demo_1}\cite{google_classroom_demo_2}.

\subsubsection{Schoology}

Schoology is also quite simple but is a little more structured than Google Classroom.

Demo video of Schoology \cite{schoology_demo}

\subsubsection{Cornerstone Learning}

Cornerstone Learning is designed for companies to provide training to their employees.

Demo videos of Cornerstone Learning \cite{Cornerstone_demo_1}\cite{Cornerstone_demo_2}.

\subsubsection{Docebo}

Docebo is a cloud-based LMS designed for training employees and customers from a company. This could be used by an institution, but it is more optimised for companies.

Demo videos of Docebo \cite{Docebo_demo_1}\cite{Docebo_demo_2}.

\subsubsection{Litmos}

Litmos is designed for companies, not institutions.

Demo videos of Litmos \cite{Litmos_demo}.

\subsubsection{TalentLMS}

TalentLMS is also designed for companies, but it feels simpler and more intuitive. 

Demo video of TalentLMS \cite{TalentLMS_demo}.


\section{Socials applications}

\subsection{Competition}

\subsubsection{Discord}

Discord is a social application where users can join servers and chat with each other. It is an application comprised mainly of real-time message threads. It also allows for video and audio calls over the Internet. Discord is not designed for delivering a structured course, although it could be possible if someone gets creative.

Demo videos of Discord \cite{discord_demo_1}\cite{discord_demo_2}\cite{discord_demo_3}

\subsubsection{WhatsApp}

WhatsApp is a real-time messaging app owned by Facebook/Meta which allows users to text over the internet. It uses mobile phone numbers to verify, authenticate and uniquely identify users. It allows for video and audio calls over the Internet.

Demo video of WhatsApp \cite{WhatsApp_demo}

\subsubsection{Messenger}

Messenger is another messaging app owned by Fackbook/Meta, which is similar to WhatsApp but instead uses a Facebook account to uniquely identify users of the application. It also connects and seamlessly links with Facebook. It also allows for video and audio calls over the Internet.

Demo video of Messenger \cite{Messenger_demo}

\subsubsection{Snapchat}

Snapchat is a messaging app which, until recently, could only be accessed on a mobile phone. It has messages which delete either after viewing once or now by default to delete after 24 hours or until the person views the message (whichever comes latest). It also allows for video and audio calls over the Internet.

Demo video of Snapchat \cite{snapchat_demo}

\subsubsection{MS Teams}

Microsoft Teams (MS Teams) is quite a powerful application that has become used by a lot of organisations and institutions, such as Cardiff University. It is designed for organisations and institutions. The main focus is on people in a team communicating with each other in many different ways, such as sharing files, real-time messaging, video/voice calls, posts and comments. MS Teams is not designed for delivering a structured course, although there would be nothing stopping someone from getting creative for this purpose.

Demo video of MS Teams \cite{ms_teams_demo}

\subsubsection{Slack}

Slack is another messaging application used as an internal messaging service for corporations and institutions. It is similar to MS Teams.

Demo video of MS Teams \cite{Slack_demo}

\subsubsection{Stack Overflow}

Stack Overflow is designed for computer scientists and software engineers to post programming and technical-related questions for the community to help and attempt to answer the problem.

Demo video of Stack Overflow \cite{Stack_Overflow_demo}

\subsubsection{Outlook (Email)}

Email has been around for the longest, originally being released in 1971 \cite{email_release}. Email is used in almost every organisation and institution, with Outlook or Gmail being the main choice of providers. It allows people to send mail over the Internet instead of posting a physical letter.

Demo videos of Outlook (Email) \cite{Outlook_demo_1}\cite{Outlook_demo_2}

\section{Learning Central + Discord}
The goal was to create a prototype of an application which combines features from LC and Discord. This was due to the inspiration of the Discord server our year group has developed over the past two-plus years. The idea was to take the VLE features such as module organisation and module resource uploads from LC and pair it with the real-time chat features from Discord.
